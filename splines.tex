\documentclass[11pt, oneside]{article}   	% use "amsart" instead of "article" for AMSLaTeX format
\usepackage{geometry}                		% See geometry.pdf to learn the layout options. There are lots.
\geometry{letterpaper}                   		% ... or a4paper or a5paper or ... 
%\geometry{landscape}                		% Activate for rotated page geometry
%\usepackage[parfill]{parskip}    		% Activate to begin paragraphs with an empty line rather than an indent
\usepackage{graphicx}				% Use pdf, png, jpg, or eps§ with pdflatex; use eps in DVI mode
								% TeX will automatically convert eps --> pdf in pdflatex		
\usepackage{amssymb}
\usepackage{amsthm}
\usepackage{mathtools}
\usepackage{parskip}

%SetFonts

%SetFonts


\title{B Spline Primer}
\author{Erik Strand}
%\date{}							% Activate to display a given date or no date

\begin{document}
\maketitle

\section{Spline Basis Functions}

\subsection{Definition}
Assume we are given $m$ knot points $\{u_0, \ldots, u_{m-1}\}$. Typically these are real numbers, though any field will do. We'll call an interval of the form $[u_i, u_{i+1}]$ the $i$th knot interval.

For $0 \leq i < m - 1$, define the order zero basis functions $N_{i, 0} : \mathbb{R} \rightarrow [0, 1]$.
\begin{equation}
N_{i, 0} (u) =
\begin{cases}
1 $ if $ u_i \leq u < u_{i+1} \\
0 $ otherwise$
\end{cases}
\end{equation}
For $1 \leq k < m - 1$ and $0 \leq i < m - k - 1$, define the order $k$ basis functions recursively.
\begin{equation}
N_{i, k} (u) = \frac{u - u_i}{u_{i+k} - u_i} N_{i, k-1} (u) + \frac{u_{i+k+1} - u}{u_{i+k+1} - u_{i+1}} N_{i+1, k-1} (u)
\end{equation}

\subsection{Basic Properties}
These functions have a number of important properties. Here are some of the most easily verifiable ones.
\begin{enumerate}
\item $0 \leq N_{i,k} (u) \leq 1$ for all $u$.
\item $N_{i,k}$ is a piecewise degree $k$ polynomial in $u$.
\item $N_{i,k}$ is nonzero only on $[u_i, u_{i+k+1}]$. 
\item On any knot interval $[u_i, u_{i+1})$, at most $k + 1$ degree $k$ basis functions are nonzero: $N_{i-k, k}$, \ldots, and $N_{i, k}$.
\end{enumerate}

\subsection{Partition of Unity}
We already know that all basis functions have a range of $[0, 1]$. It remains to show that they sum to one.


Claim: Fix any $i$ such that $k \leq i < m - k - 1$. Then for all $u$ in $[u_i, u_{i+1}]$, the sum of all degree $k$ basis functions evaluated at $u$ is 1.
\begin{proof}
As we have seen, on the $i$th knot interval only $N_{i-k}$ through $N_{i, k}$ are nonzero. From the definition of the basis functions, the following is clear.
\begin{equation}
\sum_{j=i-k}^{i} N_{j,k} (u) = \sum_{j=i-k}^{i} \frac{u - u_j}{u_{j+k} - u_j} N_{j, k-1} (u) + \sum_{j=i-k}^{i} \frac{u_{j+k+1} - u}{u_{j+k+1} - u_{j+1}} N_{j+1, k-1} (u)
\end{equation}
The first sum on the right hand side can be expanded as follows.
\begin{align*}
\sum_{j=i-k}^{i} \frac{u - u_j}{u_{j+k} - u_j} N_{j, k-1} (u) &= \sum_{j=i-k+1}^{i} \frac{u - u_j}{u_{j+k} - u_j} N_{j, k-1} (u) + \frac{u - u_{i-k}}{u_{i} - u_{i-k}} N_{i-k, k-1} (u) \\
&= \sum_{j=i-k+1}^{i} \frac{u - u_j}{u_{j+k} - u_j} N_{j, k-1} (u)
\end{align*}
This follows because $N_{i+1, k-1} (u)$ is zero, which we know because $N_{i-k+1, k-1}$ through $N_{i, k-1}$ are the only degree $k-1$ basis functions that are nonzero on the $i$th knot interval.

The second sum in (3) can be expanded in a similar way.
\begin{align*}
\sum_{j=i-k}^{i} \frac{u_{j+k+1} - u}{u_{j+k+1} - u_{j+1}} N_{j+1, k-1} (u) &= \sum_{j=i-k}^{i-1} \frac{u_{j+k+1} - u}{u_{j+k+1} - u_{j+1}} N_{j+1, k-1} (u) + \frac{u_{i+k} - u}{u_{i+k} - u_{i+1}} N_{i+1, k-1} (u) \\
&= \sum_{j=i-k+1}^{i} \frac{u_{j+k} - u}{u_{j+k} - u_j} N_{j, k-1} (u)
\end{align*}
This relies on the fact that $N_{i+1, k-1} (u) = 0$, and a simple reindexing of the summation variable.

By substituting these results into the initial equation, we see that
\begin{align*}
\sum_{j=i-k}^{i} N_{j,k} (u) &= \sum_{j=i-k+1}^{i} \frac{u - u_j}{u_{j+k} - u_j} N_{j, k-1} (u) + \sum_{j=i-k+1}^{i} \frac{u_{j+k} - u}{u_{j+k} - u_j} N_{j, k-1} (u) \\
&= \sum_{j=i-k+1}^{i} \left( \frac{u - u_j}{u_{j+k} - u_j} + \frac{u_{j+k} - u}{u_{j+k} - u_j} \right) N_{j, k-1} (u) \\
&= \sum_{j=i-k+1}^{i} N_{j, k-1} (u)
\end{align*}
We have reduced the range of summation and the degree of the summed basis functions by one. So by applying this relation recursively ($k$ more times, to be precise) we deduce that
\begin{equation}
\sum_{j=i-k}^{i} N_{j,k} (u) = \sum_{j=i}^i N_{j, 0} (u) = N_{i, 0} (u)
\end{equation}
This is by definition 1, which completes the proof.
\end{proof}

\subsection{Clamped knot vectors}
Show that you end up with one function equal to one at 0, and one function equal to one at 1.


\section{Spline curves}
\subsection{Definition}
A spline curve parametrizes a path in a vector space. Each spline curve is determined by a knot vector and a degree, which are used to construct the spline basis functions, and a set of control points in the vector space, which act as guides for the curve. 

Fix $n$ control points $\{p_0, \ldots, p_{n-1}\}$, $m$ knots $\{u_0, \ldots, u_{m-1}\}$, and a degree $k$ such that $m = n + k + 1$. (Typically the knots are real numbers, and the control points are in a real vector space such as $\mathbb{R}^2$ or $\mathbb{R}^3$. One can use a more exotic vector space, as long as its underlying field matches that of the knot vector.) Then for $u$ in $[u_{k}, u_{m-k-1}]$, define the spline curve as follows.
\begin{equation}
S(u) = \sum_{i=0}^{n-1} N_{i, k} (u) \cdot p_i
\end{equation}
We require that $u$ be in the stated interval because this is the interval on which the basis functions form a partition of unity. Alternatively, we could allow $u$ to be in $[u_0, u_{m-1}]$, and recognize that many of the properties we will examine shortly only hold for a subset of the spline curve. In either case, typically the knots are chosen so that the first and last $k$ knots are equal. For example, a typical knot vector for a degree 3 spline with 4 control points would be $\{0, 0, 0, 1/3, 2/3, 1, 1, 1\}$. We will call splines with such knot vectors clamped. Note that for clamped splines, the alternative definition mentioned above becomes equivalent to our own.

\subsection{Basic properties}
These curves have a number of important properties. Here are some of the most easily verifiable ones.
\begin{enumerate}
\item Translation and rotation invariance: to translate or rotate a spline curve, one need only translate or rotate the spline's control points.
\item $C(u_0) = p_0$ and $C(u_{m-1}) = p_{n-1}$.
\item Because the basis functions form a partition of unity, $C(u)$ is inside the convex hull of the control points for all $u$.
\end{enumerate}


\end{document}